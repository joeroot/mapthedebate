% The background section of the report should set the project into context by relating it to existing published work which you read at the start of the project when your approach and methods were being considered. There are usually many ways of solving a given problem, and you shouldn't just pick one at random. Describe and evaluate as many alternative approaches as possible. The published work may be in the form of research papers, articles, text books, technical manuals, or even existing software or hardware of which you have had hands-on experience. Your must acknowledge the sources of your inspiration. You are expected to have seen and thought about other people's ideas; your contribution will be putting them into practice in some other context. However, avoid plagiarism: if you take another person's work as your own and do not cite your sources of information/inspiration you are being dishonest; in other words you are cheating. When referring to other pieces of work, cite the sources where they are referred to or used, rather than just listing them at the end. Make sure you read and digest the Department's plagiarism document .

% In writing the Background chapter you must demonstrate your capability of analysis, synthesis and critical judgement. Analysis is shown by explaining how the proposed solution operates in your own words as well as its benefits and consequences. Synthesis is shown through the organisation of your Related Work section and through identifying and generalising common aspects across different solutions. Critical judgement is shown by discussing the limitations of the solutions proposed both in terms of their disadvantages and limits of applicability.

\chapter{Background}
\label{background}

From its early forums through to the 'social web' of today, the Internet has served as a continually expanding platform for discussion. The result has been an explosion in the amount of readily available, computer-formatted textual opinion. With this growth has come an increasing desire to computationally understand the wealth of opinion now so easily accessible. Combining elements of linguistics, natural language processing and machine learning, this field of exploration has come to be known as \emph{opinion mining} or \emph{sentiment analysis}. In the following chapter we will first briefly examine Twitter as a backdrop to our discussion on sentiment analysis. We will then go on to explore the general problems posed by sentiment analysis along with the common approaches and solutions taken in addressing them. In sections \ref{background:discovering_opinion} - \ref{background:topic_extraction} we will discuss in detail the areas and methods of sentiment analysis which will bear relevance to this project's Twitter-based setting. In section \ref{background:emotion} we will explore emotion in general, particularly looking at its scope and ways of classifying it. Finally in section \ref{background:tools} we shall discuss our project's choice of programming languages and tools.

\section{Twitter}
\label{background:twitter}

Twitter is a social-networking web-service. It enables users to post and read 140 character messages known as \emph{tweets}. A user's \emph{timeline} serves as a publicly viewable history of their tweets. Furthermore if someone chooses to \emph{follow} another user, they will be notified of changes to that user's timeline. This simplicity has seen Twitter's user-base rapidly expand, with over 200 million active users today. From football transfers to revolutions Twitter has become the go-to service for spreading news quickly and efficiently.

Since its launch in 2006, certain protocols have emerged from within the Twitter community. These have been embraced by Twitter, enabling it to serve not only as an efficient platform for spreading news, but also as a rich and sophisticated medium for conversation. Notable protocols include:

\begin{description}
	\item [Hashtags] enable users to tag their tweets with any word or combination of characters they deem appropriate. Although this may seem basic at first, through common hashtags, it enables users to take part in a community-wide discussion. For example, during the recent voting reform referendum, the hashtags '\emph{\#yes2av}' and '\emph{\#no2av}' were used to form a debate on the strengths and weaknesses of the Alternative Vote. 
	\item [Mentions] allow users to reference other users in their tweets. Furthermore if a user is mentioned in a tweet, Twitter will notify the mentioned user. Through this, Twitter users can take part in a direct conversations with one or more other users. For example, if we wanted to ask Stephen Fry a question, we could tweet '\emph{what are you eating for breakfast @stephenfry?}'.
	\item [Re-tweets] give users the ability to re-post other users' tweets in their own timeline. This simple feature has had a significant impact on Twitter's ability to facilitate the rapid spread of news. For example in 2009 when the US Airways flight 1549 crash landed in the Hudson river, rapid re-tweeting of an amateur photo meant the news broke on Twitter far earlier than it did within the media at large. This has continued to be true for many more notable events such as the recent North-African revolutions.
	\item [Links] have always been the popular subject of tweets, however the introduction of link-shorteners has changed the way in which they are posted. In freeing up characters by shortening a URL, users now have the option to describe or comment on the link they are tweeting. This has enabled users to engage in deeper conversation on content they have viewed online, and has neatly allowed Twitter's viral nature to better merge with its community's desire for debate.
\end{description}

Through Twitter's RESTful API \footnote{RESTful APIs allow developers to retrieve, modify, create and delete data by making get, post and delete HTTP requests to specified web addresses.}, this rich resource of live news and debate will serve as the project's main data source.

\section{Sentiment analysis}
\label{background:sentiment_analysis}

Sentiment analysis as a field, is the exploration of how we can computationally understand opinions expressed within a body of text. In order to do this, we must first define a computational structure for expressing opinions. In general \cite{Liu:2010tm} this is done by breaking an opinion down into four parts. Firstly we must determine the opinion's focus of discussion, also known as its \emph{topic}\footnote{This is more commonly referred to within literature as an opinion's \emph{feature}, however to avoid later confusion with the machine learning term, we will use the term \emph{topic}.}. This in practise can encompass anything from Government policy to mobile phone battery life. Often opinion is not necessarily that of the author, but of a referenced person or group, therefore it is important to determine the opinion's \emph{holder}. Along with this it is also often necessary to determine the \emph{time} at which the opinion was expressed. Finally, we hope to \emph{classify} (or in some cases quantify) the opinion which has been passed. Leading research \cite{Pang:2002tu,Turney:2002vv} has typically focussed on discrete classification, such as deciding whether an opinion is positive, negative or neutral. A fifth \emph{object} component is sometimes introduced for larger documents, which serves as an identifier for related topics. For example, within a phone review the majority of opinions may share the same object, in this case the phone, but focus on different topics such as battery life or call quality. 

How do we computationally discover opinions and identify their parts? In general the approach can be loosely split into two components, \emph{sentence-level classification} and \emph{document-level classification}. Sentence-level classification determines whether a sentence expresses an opinion along with classifying that opinion if it exists. Furthermore if an opinion is found, sentence-level classification will try to determine its topic, holder and the time at which the opinion was cast. Document-level classification goes on to collate the sentence-level results, in order to form a general description of the document's sentiment. Both these approaches draw heavily upon machine learning techniques. It is important to note here however, a core criticism of the field. Linguists such as Chomsky \cite{norvig} observe that rather than truly trying to understand and define the semantics of sentiment, the field takes a heavily statistical approach. This means that rather than determining sentiment by forming a semantic conclusion, the field uses a limited linguistic foundation to predict sentiment based upon experience. Nonetheless, redefining natural language processing and sentiment analysis is not within the scope of this project, and we shall proceed with the field's successfully tried and tested approaches.

As we shall discuss in more detail in chapter \ref{subjectivity}, only sentence-level classification is relevant to this project. Furthermore, methods for determining an opinion's holder and time are unnecessary and will not be discussed here. The remainder of this section will instead focus on the three relevant topics from within sentence-level classification. Firstly we shall explore what exactly an opinionated sentence is and how we can computationally determine this. Next we will look at common approaches to classifying sentiment, before finally examining how we determine the topic of an opinion. Before this however, we shall briefly outline the concepts and methods of \emph{supervised learning} as this shall form the core for each of our classification problems.

\section{Supervised learning}
\label{background:supervised_learning}

Supervised learning is a task within machine learning which infers a function from a set of training data. This approach is well suited to classification problems, and in our case is particularly relevant in discovering opinion and determining polarity. Thus, the remainder of this section will discuss supervised learning with respect to the classification of sentences. 

\subsection{Defining the problem}

In both discovering opinion and determining polarity we want to find an approximate hypothesis function $h$, for an actual function $c$, where $c$ would perfectly perform either task. Both functions will map an input sentence $s\in S$, to a discrete classification $o\in O$, where $S$ is the set of all possible sentences and $O$ is the set of all possible classifications, for example $O=\{positive, negative\}$, such that:

\begin{equation}
	h \approx c:S \rightarrow O
\end{equation}

In order to find our best fit hypothesis function $h$, we will first need to determine a set of \emph{features} for our sentences. Within machine learning, features are the attributes which best describe and discriminate our input data when trying to classify it. For example if we are trying to learn a function to decide whether we should play tennis or not, features might include humidity and sunlight. In essence we want to identify a list of the most useful features $f_1, f_2,\dots,f_n$ for our sentences, such that:
	
\begin{equation}
	h \approx c : \langle f_1, f_2,\dots,f_n \rangle \rightarrow O
\end{equation}

Once a set of features has been chosen we can approximate $h$ by training it. In order to find the perfect hypothesis function for classifying subjective functions, i.e. $h=c$, we would require knowledge of every single possible sentence along with its correct classification. Clearly we could never produce the set of all possible sentences, let alone determine every sentence's classification. Instead, we select a sample of training sentences $T \subseteq S$, and manually \emph{label} each sentence $t \in T$ with a classification $l \in O$. This is our \emph{training data} $D$, such that:

\begin{equation}
	D = \{(t, l) : \forall t \in T \textit{ there exists a manually labelled classification $l$} \}
\end{equation}

Given this training data we can now determine as accurate a hypothesis function as possible for classifying \emph{all} sentences. There are numerous, largely statistical methods for training our hypothesis function. Each brings their own positives and negatives, and there has been extensive research \cite{Pang:2004us} into which methods perform best for opinion based classification. We will discuss the most appropriate methods, features and training data for each classification problem in their respective parts.

\subsection{Naive Bayes Classifier}

The \emph{Naive Bayes (NB)} classifier is a probabilistic supervised classifier \cite{Rish:2001vu}. Using Bayes' theorem, it statistically predicts an inputs most likely classification based upon previous experience. Thus given an input represented as the feature set, $f_1, f_2,\dots,f_n$, a Naive Bayes classifier will essentially classify it with label $o \in O$, according to:

\begin{equation}
	\argmax_{o \in O} \Pr(o | f_1, f_2,\dots,f_n)
\end{equation}

In order to be able to predict this probability, the classifier learns from its training data, which is also represented as a series of feature sets. 

Within sentiment analysis literature \cite{Pang:2008wj,Liu:2010tm}, Naive Bayes classifiers are often experimented with, and unlike in more genral applications \cite{ChihWeiHsu:2002gr}, it often performs strongly. Naive Bayes classifiers also have the additional advantage of requiring only a limited amount of training data in order to be able to perform well.

\subsection{Support Vector Machines}

Unlike Naiver Bayes classifiers, \emph{Support Vector Machines (SVM)} are non-probabilistic, binary classifiers. Given a set of training data presented as feature sets, $f_1, f_2,\dots,f_n$, SVMs hope to find a hypothesis function which best divides the training data into its two respective classes \cite{Lin:2004th}. In essence, this is done by plotting the features in $n$ dimensions, before trying to find the hyperplane which "best" separates the features into their respective classes. In finding the "best" hyperplane, SVMs look for the hyperplane with the largest $n$-dimensional distance between it and it's nearest training example.

As Support Vector Machines present themselves as binary-classifiers, they need to be adapted in order to be able to classify more than two labels. This is referred to as multi-class classification. As outlined by Hsu et al., there have numerous attempts at implementing multi-class classification using SVMs, with each presenting its own benefits. As Caruana et al. observe, typically SVMs outperform NB classifiers, and within sentiment analysis \cite{Pang:2008wj,Liu:2010tm} in particular, Support Vector Machines often perform strongest.

\subsection{Un-supervised learning}

Un-supervised learning is a fairly general term, often used to describe approaches to learning in which there is no labelled data to draw upon on. Typically this means that data is instead clustered based upon attributes known or unknown, to the programmer. As a result of no core underlying approaches, un-supervised learning is difficult to define. However, within this project it shall refer to approaches which use no training data, and in particular those which apply no supervised techniques.

\section{Discovering opinion}
\label{background:discovering_opinion}

In general opinion manifests itself either \emph{explicitly} through \emph{subjective} sentences and phrases, or \emph{implicitly} through \emph{objective} sentences and phrases. An objective sentence expresses factual information, whilst a subjective sentence expresses a mental or emotional state, such as a sentiment or belief. A subjective sentence such as, "\emph{I love the NHS, it's bloody marvellous}", explicitly states an opinion. Similarly however, a sentence such as "\emph{Lost my job due to recent Coalition cuts}" although objective, could also be considered an implicit opinion. This clearly poses a difficult challenge for classification, and as Mihalcea et al. \cite{Mihalcea:2007uh} note, it is one which "has often proved to be more difficult than subsequent polarity classification". As observed by Liu \cite{Liu:2010tm} however, opinionated sentences tend to be a subset of subjective sentences. Due to this, the approaches for classifying them are alike and the terms are taken as interchangeable. This is referred to as \emph{subjectivity classification}.

Subjectivity classification is typically achieved through a mix of supervised and unsupervised learning. In general, unsupervised learning is used to bootstrap a relatively small but accurate training set. The bootstrapped training set is then utilised to train a classifier. Numerous feature choices have been proposed for training subjectivity classifiers. We shall first examine some of the the more commonly used features, as discussed by Wiebe et al. \cite{Wiebe:1999cj}:

\begin{description}
	
	\item [Adjectives] tend to be strong indicators of subjectivity, often serving as descriptions or qualifications of opinion. For example the adjectives in, "\emph{the coalition cuts are \textbf{harsh} but \textbf{necessary}}", are clear indications of subjectivity. As Wiebe et al. \cite{Wiebe:1999cj} observe a simple binary feature alone, noting the appearance of one or more adjectives, results in a classification accuracy of 56\%. 
	
	\item [Adverbs] modify verbs, adjectives and phrases, for example "\emph{they \textbf{usually} get things right}". Their presence is often an indicator of subjectivity, and although not as useful as adjective presence, their inclusion as a binary feature further improves classification rates. Wiebe et al. \cite{Wiebe:1999cj} suggest a binary feature noting the presence of any adverb other than \emph{not}.
	
	\item [Pronouns] are substitutions for nouns, for example \emph{it} in place of an object. They are often minor indicators of subjectivity, and have been shown to marginally improve classification accuracy when included as a binary feature.
		
	\item [Adjective orientation and gradabilty] tend to be further indicators of subjectivity. Essentially orientation notes whether an adjective encodes a desirable (e.g \emph{beautiful}) or undesirable (e.g. \emph{ugly}) state. The gradability of an adjective denotes the relative extent to which an adjective varies in strength from the norm. For example \emph{"small"} and \emph{"large"} have high gradability. As shown by Wiebe et al. \cite{Wiebe:2000tk}, the presence of polarised, gradable adjectives is a strong measure of subjectivity and a useful feature.
	
\end{description}

Wiebe et al. \cite{Wiebe:1999cj} observed that using the first three features, coupled with a feature noting cardinal numbers presence, resulted in classification rates of 71.2\%. 

But how can we identify these features within a sentence? Adjectives, adverbs and pronouns are all known as \emph{parts of speech (POS)}. A word's POS can take on one of eight roles within a sentence: \emph{verb}, \emph{noun}, \emph{pronoun}, \emph{adjective}, \emph{adverb}, \emph{preposition}, \emph{conjunction} and \emph{interjection}. A word's part of speech is often determined by its position within the sentence. For example "\emph{love}" can be a noun or a verb, dependant upon the context in which it is used. Below is an example of a sentence whose words have been \emph{tagged} with their POS:

\begin{equation}
	\underbrace{{\rm She}}_{\rm pron.} \overbrace{{\rm likes}}^{\rm verb} \underbrace{{\rm big}}_{\rm adj.} \overbrace{{\rm snakes}}^{\rm noun} \underbrace{{\rm but}}_{\rm conj.} \overbrace{{\rm I}}^{\rm pron.} \underbrace{{\rm hate}}_{\rm verb} \overbrace{{\rm them.}}^{\rm pron.}
\end{equation}

\subsection{Part of speech tagging}
\label{background:pos}

Given a phrase or sentence, \emph{part of speech tagging} computationally determines each word's POS. This can be done in variety of ways. Typically basic implementations use a lexicon of words with their appropriate tags, or a more advanced dictionary such as WordNet\footnote{WordNet is a detailed dictionary with additional levels of detail describing the semantic interlinking between words. It will be used throughout this project and shall be discussed in more detail in section \ref{emotion:wordnet}.}. In general these implementations are naive and often simply return a list of possibilities. More intuitive techniques tend to use machine learning to recognise patterns, or are built with a set of linguistic rules. We will discuss the merits of these techniques and their implementation in more detail in chapter \ref{subjectivity}. With a fully tagged sentence it is now possible to build a feature set based upon the relevant parts of speech.

\subsection{Use of supervised techniques}

As noted in our discussion of features, some adjectives are more useful in classifying subjectivity than others. Determining these adjectives, and in this case their polarity, would prove tedious if carried out by hand. Instead, Wiebe \cite{Wiebe:2000ub} suggests a supervised approach using a small set of a hundred or so seed words, such as \emph{good} and \emph{bad}, and a large corpora of text. The corpora is examined for conjunctions, such as "\emph{handsome} and \emph{smart}", and disambiguations such as "\emph{smart but cruel}". When a seed word is found within either scenario, its fellow word's polarity can be inferred. For conjunctions, if one of the words is known as positive, then the unknown word is likely to be positive also. The converse holds for disambiguations, where the unknown word is inferred to be the opposite of the known word. This technique enables the rapid building of a polarised adjective lexicon. It is particularly useful in domains which assign their own meaning to adjectives, for example \emph{sick} is often a positive adjective within youth culture.

Building a training set significant enough for accurate subjectivity classification can often be time consuming. Liu \cite{Liu:2010tm} and Akkaya et al. \cite{Akkaya:2009ww} describe a supervised method for bootstrapping an initial training set. A high precision, low recall rule based classifier, as originally proposed by Wiebe and Riloff \cite{Wiebe:2003wa}, is used to build a small training set from a large corpora. The classifier does this by identifying strong and weak subjective clues within a sentence. If there are two or more strong subjective clues the sentence is classified as subjective. In order to determine objectivity, the sentences on either side are taken into account. If between them neither contain more than one strong and two weak clues, along with no strong clues in the analysed sentence, the sentence is considered objective. If the conditions for subjectivity and objectivity are not met, the classifier leaves the sentence unclassified. The use of supervised methods such as this and the lexicon builder described above are typical within the field. They provide simple and efficient ways of optimising the overall training process.

\subsection{Present research and issues}

Recent literature has also explored numerous improvements to the classic algorithm as described above. One such improvement of notable effect is \emph{subjectivity word sense disambiguation (SWSD)}, originally presented by Wiebe et al. \cite{Wiebe:2006te}, and further refined by Akkaya et al. \cite{Akkaya:2009ww}. SWSD tries to reduce the misclassification of objective words, and thus possibly the sentence, as subjective. These false hits often occur as a result of assuming that if a word exists within a subjective lexicon, it is being used in subjective sense. For example, \emph{pain} is often used subjectively, however within, \emph{early symptoms include body pain}, \emph{pain} is used in an objective sense. SWSD attempts to eliminate this source of error. A subjective lexicon of words is built, and for each of its words, a classifier is trained. Given a potentially subjective word within a sentence, the classifier will label the word's sense as objective or subjective. The classifier is trained using a corpora of sentences whose subjective words have been labelled as either subjective or objective. The classifier is then used to ensure that all subjective words are used in their subjective sense. Using SWSD within subjectivity classification, Wiebe et al. \cite{Akkaya:2009ww} noted a 24\% reduction in error against a classifier using the regular subjectivity lexicon when looking for subjective words. 

Subjectivity classification is a well researched field, however current methods do pose problems. As is typically the case within supervised learning, the classifier's ability is significantly influenced by how representative its training set is of the input domain. Subjectivity classification, along with many other natural language approaches, is often extremely sensitive to the type of content with which it has been trained. This means that if one wants to build a subjectivity classifier for political speeches, the training corpora should be built from similar content, not for example from movie reviews. No fixed approach has been developed for this, and it is an issue we shall have to contend with during our implementation in chapter \ref{subjectivity}. 

\begin{comment}
	
Subjectivity word-sense disambiguation \cite{Akkaya:2009ww}

- typical approaches rely on lexicon of words
	- these are usually word lists rather than meanings (or senses)
	- can lead to false hits i.e. word assumed to imply s, when really it is o
- thus Subjectivity Word Sense Disambiguation
	- labels clue words as subjective or objective sense
	- more feasible than full word-sense
- use SVM
- use bootstrapping to create a training set

- since sentences often contain multiple subjective expressions, expression level classification is more informative than sentence-level classification

Word sense and subjectivity \cite{Wiebe:2006te}
- propose that there are motivations for separate classifiers, one each
- top of page 6 makes very good point about a word typically being subjective, being objective in a subjective sentence

Learning subjective adjectives from a corpora \cite{Wiebe:2000ub}

Effects of adjective orientation \cite{Wiebe:2000tk}
- use word conjunctions to find positive/negative adjectives
	- e.g. would say corrupt and brutal, not corrup or brutal

\end{comment}


\section{Classifying opinion}
\label{background:sentiment_classification}

An opinionated sentence can express a diverse range of sentiment, and classifying this can prove difficult. Sentiment can be classified in numerous ways, for example "\emph{I liked the tone of his speech, however I am uncertain of the proposals within it}", could be interpreted in any number of ways. At a phrase level, we might consider the first part to express some form of delight, while the latter expresses distrust. Of course, to a certain extent these are subjective, and more detailed emotional labels shall be discussed in section \ref{background:emotion}. A more broad classification might classify the first part as positive and the second part as negative. Developing methods for labelling a sentence's polarity has served as a focus for much of the research into classifying opinion. This field is referred to as \emph{sentiment classification}.

But how do we determine sentiment? At first this may seem simple. For example "\emph{I \textbf{love} the EU}" would typically be classified as positive, whilst "\emph{I \textbf{hate} the EU's decentralisation of power}" would be negative. Clearly \emph{love} and \emph{hate} are strong indicators of polarity. Basic methods for classifying sentiment simply check whether any of the words within the sentence exist within a pre-defined polarity lexicon, and classify accordingly. If we explore increasingly complex phrases however, the problem becomes far less simple than simply identifying polarising words. Understanding the scope of negation can present challenges. For example the negative in "\emph{not nice}", simply negates its neighbour, whilst in "\emph{no one thinks that its good}", the ensuing negation spans the phrase. In certain scenarios negation words can even strengthen polarity, such as "\emph{not only good but amazing}". Issues of word sense, similar to those discussed in section \ref{background:discovering_opinion}, present further problems. For example "the National \emph{Trust} may waste money" conveys an opinion which expresses the polar opposite of trust. The domain of the sentiment being expressed can also effect polarity. "\emph{Go read the book}" may be considered positive within a book review, however for a film it is generally see as negative.

At its heart sentiment classification poses a significant linguistic challenge, and the approaches vary as a result of it. They can be broadly split into two approaches however, supervised and unsupervised. Unsupervised methods propose that sentiment can be understood by analysing its linguistic form. By understanding the rules which allow sentiment to be expressed, we should be able to both identify and understand it within a sentence. Supervised methods suggest that the complexities of language make unsupervised methods too specific and difficult to identify. Instead it hopes to make use of machine learning's supervised techniques in order to better classify sentiment. We will explore and contrasts these two methods. In particular, we will focus upon the unsupervised approach put forward by Turney \cite{Turney:2002vv}, and the supervised approach proposed by Pang et al. \cite{Pang:2002tu}.

\subsection{Unsupervised sentiment classification}

Turney \cite{Turney:2002vv} suggests a two part approach to supervised sentiment classification. As discussed when exploring subjectivity in section \ref{background:discovering_opinion}, adjectives tend to be a significant grammatical structure through which sentiment is expressed. Thus, Turney proposes extracting phrases containing adjectives and whose structure indicates an expression of sentiment. Given a sentence, we tag its parts of speech, before extracting any two-word phrases whose structure can be found within the following linguistic patterns:

\begin{table}[h]
	\caption{Extraction patterns for identifying opinionated two-word phrases}
	\label{background:patterns}
	\centering
	\begin{tabular}
		{ l | l l l }
		Rule & First word & Second word & Third word (\emph{not extracted}) \\ \hline
	  1. & JJ & NN, NNS & anything \\
		2. & RB, RBR, RBS & JJ & not NN, not NNS \\
		3. & JJ & JJ & not NN, not NNS \\
		4. & NN, NNS & JJ & not NN, not NNS \\
		5. & RB, RBR, RBS & VB, VBD, VBN, VBG & anything \\
	\end{tabular}
\end{table}

Once these phrases have been identified, we can then determine their sentiment's polarity. This is done by first selecting two words commonly associated with strong positive and negative sentiment. Turney suggests \emph{excellent} and \emph{poor} as the benchmark words for positive and negative polarity. This is largely due to their prevalent use within reviews as descriptions for high and low ratings. In order to calculate a phrases sentiment, we attempt to measure the association between it and benchmark's words. Co-occurence between two words is calculated using their \emph{Pointwise Mutual Information (PMI)}, defined as:

\begin{equation}
	\operatorname{PMI}(word_1, word_2) = \log_2 \left(\frac{\operatorname{p}(word_1 \textit{ \& } word_2)}{\operatorname{p}(word_1)\operatorname{p}(word_2)}\right)
\end{equation}

Where $\operatorname{p}(word_1,word_2)$ is the probability that $word_1$ and $word_2$ co-occur within a corpora, and $\operatorname{p}(word)$ is the probability that $word$ occurs. Now that we have definition for PMI, we can define the \emph{semantic orientation (SO)} of a $phrase$ as:

\begin{equation}
	\operatorname{SO}(phrase) = \operatorname{PMI}(phrase, "excellent") - \operatorname{PMI}(phrase, "poor")
\end{equation}

The resulting semantic orientation is a measure of a phrase's sentiment. An SO larger than 0 denotes positive polarity, while an SO less than zero indicates negative polarity. Thus, a sentence's overall polarity is simply the average of its phrases' SO. This approach to supervised sentiment classification has proven effective across a variety of review domains. Turney reports an impressive 80\% when classifying bank reviews and an even better 84\% accuracy for automobile reviews. He does note however, that movie reviews present a challenge for his supervised approach, reporting an accuracy of 65.83\% within the movie domain. Nonetheless, across domains Turney reports classification rates of 74.39\%, demonstrating the strong potential which lies within unsupervised methodologies.

\subsection{Supervised sentiment classification}

Shortly after Turney published his paper on supervised approaches \cite{Turney:2002vv}, Pang et al. \cite{Pang:2002tu} put forward a counter paper. This addressed the potential of supervised learning within the same domain of internet reviews as Turney's original paper. At its core, Pang et al. address the issue that often sentiment can be expressed in very subtle ways. For example, "\emph{How could anyone sit through this movie?}" does not express negative opinion in any readily apparent way. Essentially the proposition put forward by Pang et al. is that the nuanced structures through which we express opinion are too vast and varied. They cannot simply be whittled down into a simple set of rules, and rather, we should look to experience to guide our classification.

As with any supervised problem, the learning experience is largely guided by our choice of features. Before we examine these, it is important to introduce the concept of \emph{n-grams} and how they work as features. For example, if we use unigram feature set, there is a feature for every possible word. A feature set this large is unnecessary however, as the only words which will be important in classification are those we encounter in training. Thus we build a feature set from the words we encounter when training. If our training set only contained "\emph{I love the NHS}", we would have the following feature set for classification $\langle f_{I}, f_{love}, f_{the}, f_{NHS} \rangle$. Alternatively if we used bigrams (2 word phrases), we would have a feature set $\langle f_{(I,love)}, f_{(love,the)}, f_{(the,NHS)} \rangle$. But what values do we assign to these features when given a sentence to classify? Pang et al. experiment with two options:

\begin{enumerate}
	\item{\emph{Term presence} denotes whether the n-gram phrase that a feature represents occurs within our sentence. For example, using the unigram and bigram feature sets above, and given a sentence "\emph{I hate the NHS}", we would have the following feature sets:
	\begin{eqnarray} 
		& \langle f_{I}, f_{love}, f_{the}, f_{NHS} \rangle = \langle true, false, true, true \rangle \nonumber \\
		& \langle f_{(I,love)}, f_{(love,the)}, f_{(the,NHS)} \rangle = \langle false, false, true \rangle \nonumber
	\end{eqnarray}
	}
	\item{\emph{Term frequency} denotes how frequently each feature's n-gram phrase occurs within our sentence. For example, using the unigram and bigram feature sets above, and given a sentence "\emph{I hate the NHS, but I love my GP}", we would have the following feature sets:
	\begin{eqnarray} 
		& \langle f_{I}, f_{love}, f_{the}, f_{NHS} \rangle = \langle 2, 1, 1, 1 \rangle \nonumber \\
		& \langle f_{(I,love)}, f_{(love,the)}, f_{(the,NHS)} \rangle = \langle 1, 0, 1 \rangle \nonumber
	\end{eqnarray}
	}
\end{enumerate}

Pang et al. also experiment with appending POS tags to the end of each word, thus distinguishing between their possible uses. In order to handle negation, any words between a negative word such as \emph{not} and the next punctuation mark are tagged with a $NOT$. For example "I do not like the NHS" would result in a feature set $\langle f_{I}, f_{do}, f_{not}, f_{NOT-like}, f_{NOT-the}, f_{NOT-NHS} \rangle$. 

The different feature sets were tested within the movie review domain. The presence feature set for unigrams performs strongest in their experiments with an accuracy of 82.9\%. The combination of unigrams and bigrams sees a marginal drop in accuracy to 82.7\%. Interestingly POS tags also have a slight negative effect on accuracy, seeing it drop to 81.9\% when coupled with a unigram presence feature set. In domains where the expression of sentiment is subtle, supervised approached have a clear benefit over their unsupervised counterparts. However, supervised learning requires one to build a training set, which can often prove time consuming. Furthermore its understanding of sentiment is based upon experience, thus it could never really explain why it reached its decision. Deciding which approach is better is difficult, and we shall explore this is more detail in section \ref{polarity}.

\subsection{Present research and issues}

Recent research has focussed on how combinations of supervised and unsupervised learning can be used to improve classification rates. Essentially these improvements have hoped to introduce greater linguistic detail into the supervised approach described by Pang et al.. In the following section we shall provide a general overview of two improved methodologies put forward by Wilson et al. \cite{Wilson:2005tt} and Benamara et al. \cite{Benamara:2007wz}.  We shall explore these approaches in greater detail in section \ref{polarity}.

Although Wilson et al. \cite{Wilson:2005tt} acknowledge the need for elements of supervised learning, they observe that the sentence-level approach put forward by Pang et al. is to general. Instead they propose that to truly understand sentiment, we must approach it at a phrase level. The main motivation behind this is the common misclassification of \emph{clue} words as polar, when the sense in which they are being used means they are in fact neutral. This problem is of particular relevance to the supervised approach discussed above. The method put forward by Pang et al. essentially creates a lexicon of polar words during training and later uses them as clue's for classifying polarity. As mentioned in our introduction to opinion classification, this can lead to words being taken out of context to classify neutral statements as polar. Wilson et al. propose a two step solution to this. The first step identifies all clue phrases within a sentence, before using a supervised approach to classify each one as polar or neutral. The polarity of each polar phrase is then disambiguated to give it an overall classification of either \emph{positive}, \emph{negative} or \emph{both}. Not only does this approach provide a more rigorous framework for sentiment classification, unlike the methods put forward above it also acknowledges the potential neutrality of phrases within a sentence.

Alongside the influential research into phrase-level sentiment by Wilson et al., other prominent research has focussed on measuring sentiment strength. Benamara et al. \cite{Benamara:2007wz} highlight the important role of adverbs as measures of opinion. These adverbs are known as \emph{adverbs of degree}. Within this subset of adverbs, five clear classifications can be noted:

\begin{enumerate}
	\item \emph{Affirmation} adverbs such as \emph{certainly} and \emph{absolutely} strengthen adjectives.
	\item \emph{Doubt} adverbs such as \emph{possibly} and \emph{seemingly} weaken adjectives.
	\item \emph{Strong intensifying} adverbs such as \emph{exceedingly} and \emph{extremely} strengthen adjectives.
	\item \emph{Weak intensifying} adverbs such as \emph{barely} and \emph{scarcely} weaken adjectives.
	\item \emph{Negation/minimising} adverbs such as \emph{hardly} and \emph{rarely} invert or neutralise adjectives.
\end{enumerate}

Using a lexicon containing adverbs of degree and their appropriate classification, all unary and binary adverb adjective combinations are found. A unary combination has the form $\langle adverb \rangle\langle adjective \rangle$, whilst a binary combination has the form $\langle adverb_i, adverb_j \rangle\langle adjective \rangle$. Each adjective in the matching phrases has its polarity strength adjusted according to the classification of the adverbs which proceed it. For unary combinations the score is a product of the adjective and adverb strengths. For binary combinations, the strength of $\langle adverb_j \rangle\langle adjective \rangle$ is calculated first as if it were a unary combination, before calculating the strength combination of the resulting score and $adverb_i$. Benamara et al. report results almost on par with human strength classification, highlighting the proposed method as not only viable but effective.

Although significant improvements have been made within the field, sentiment classification is still far from perfect. Many of its problems have been reduced in size, however they have not been eradicated. One could argue that this is largely due to the statistical nature of supervised learning, and clearly the field still has a lot to learn from linguistics. Most importantly to this project however, is the fact that little research has explored beyond the confines of polarity and into the realm of emotion. We shall explore the field's limited approach to the classification of emotion in more detail later, in section \ref{background:emotion}.

\begin{comment}

Using adverbs for grading - Benamara et al. \cite{Benamara:2007wz} 

Subjectivity does not necessarily imply a polarised opinion

Broad overview in \cite{Pang:2008wj}

Phrase level sentiment analysis \cite{Wilson:2005tt}

Movie reviews, thumbs up/down \cite{Turney:2002vv} \cite{Pang:2002tu}

\end{comment}

\section{Topic extraction}
\label{background:topic_extraction}

\begin{comment}
	2 pass system
		- first look for topic
		- then measure sentiment accordingly
		- Ounis et al. \cite{Macdonald:2007vu}
		- Hurst and Nigam \cite{Nigam:2004ur}, \cite{Nigam:2004vw}
\end{comment}

\section{Sentiment on Twitter}
\label{background:sentiment_on_twitter}

With the recent and rapid growth of Twitter has come an interest in understanding the sentiment expressed on it. Although at its heart an issue of sentiment analysis, Twitter's constraints and protocols pose new and different issues for current approaches. Literature is still limited, and solutions to the problems within it are varied. In this section we will focus on some of the more prominent approaches. In particular we will outline the framework proposed by Barbosa and Feng \cite{Barbosa:ws}, whilst looking at some of the innovative improvements and observations put forward by Go et al. \cite{Go:2009ut} and Bermingham et al. \cite{Bermingham:2010vh}.

Barbosa and Feng \cite{Barbosa:ws} propose a two stage sentiment analysis framework. Firstly the subjectivity of a tweet is determined, and if subjective, the tweet's sentiment is then classified. This framework bares many similarities to sentence-level sentiment classification, however the approach within each stage is in many ways very different. Particular emphasis is placed upon the need for strong subjectivity detection. There is a lot of \emph{noise} on Twitter through adverts and spam accounts, thus it is important to filter this out if we ever hope to obtain an accurate overview. A typical noisy tweet might be:

\begin{quote}
	\emph{Get a FREE \$500 Starbucks Gift Card >> Special Online Offer ...: Starbucks is celebrating its first forty years ... http://bit.ly/iepyV5}
\end{quote}

Barbosa and Fang propose some previously unconsidered features for helping distinguish noise from subjective tweets. As evident from analysing the tweet above, Barbosa and Fang note that \emph{link presence} and \emph{uppercase letter frequency} serve as particularly useful subjectivity clues. A novel approach is taken to training the subjectivity classifier using existing online Twitter sentiment classifiers. Subjective tweets are scraped from three such sites, and any tweet appearing as subjective in all three is added to the training data. They report that although this can lead to slight bias, it serves as an effective bootstrapping method. 

Barbosa and Fang take an entirely supervised approach to polarity classification using many of the features discussed in section \ref{background:sentiment_classification}. Uppercase letter frequency again proves particularly useful, along with a feature for \emph{good emoticons}. An emoticon is a text-character face expressing an emotion, for example happy is commonly represented as \texttt{:)} while sad is \texttt{:(}. Barbosa and Fang note significant improvements both in subjectivity and sentiment classification when using tweet-based features, as opposed to the typical approaches described in sections \ref{background:discovering_opinion} and \ref{background:sentiment_classification}. Using unigrams alone for sentiment classification, Barbosa and Fang report an error rate of 44.5\%, whilst the introduction of Twitter based features reduces this to 25.1\%. Although far from perfect, the improvements are notable, and suggest that a better understanding of the intricacies of Twitter could lead to further improvements.

Interestingly recent work by Bermingham and Smeaton \cite{Bermingham:2010vh} suggests that further linguistic detail when building a feature set in fact harms classification rates. Rather than using POS tagging or larger n-grams, they note that features such as link presence and punctuation mark frequency serve as far better discriminators for subjectivity and polarity. They report accuracy rates of 74.85\%, which are strikingly similar to those achieved by Barbosa and Fang. 

Building a training set for Twitter can prove difficult due to the need for large data sets. There is an extraordinary diversity of structure, language and grammatical approach on Twitter, thus a large training set is necessary if we hope to be able to accurately classify its broad range of opinion. Further to the innovative approach taken by Barbosa and Fang, Go et al. \cite{Go:2009ut} suggest a further innovative technique for quickly building a large data set. By searching Twitter for all tweets containing positive and negative emoticons, Go et al. were able to quickly assemble list of polarised opinion. This method for building a training set proved remarkably successful, and simply using unigrams as features, they report an accuracy rate of 82.2\%. 

Although literature regarding sentiment analysis on Twitter is limited, there have been significant advances in accuracy. Interestingly, as Bermingham and Smeaton observe, detailed linguistic features seem to be of little benefit when classifying subjectivity and polarity. However, as noted by Go et al., the size of the training set seems to have a marked effect on classification accuracy. Clearly Twitter poses many new challenges for sentiment analysis, and although progress has been made, more in depth research is needed before the best approaches can be truly identified.

\begin{comment}

Understanding sentiment on Twitter however, requires slight changes to our approach. We will assume that opinion expressed within a tweet is the user's own, and that the time at which a tweet is posted reflects the time at which any opinion in it was cast. This leaves us with three core problems. First we must determine whether a tweet contains an opinion, and if so, we must be able to both classify it and determine its topic. Research into Twitter-targeted sentiment analysis is limited, however, due to the length constraints placed upon tweets, aspects of sentence-level classification still bare relevance. Furthemore work such as Bermingham' \cite{Bermingham:2010vh} and Barbosa's \cite{Barbosa:ws} highlight some key variations in approach to sentiment analysis on Twitter.

\cite{Barbosa:ws} - suggests 2-step framework, subjectivity detetction then classification
\cite{Barbosa:ws} - additional features from tweet syntax

Robust sentiment detection \cite{Barbosa:ws}
	- good overview

Identifying themes/sentiment \cite{KumarPal:2010fd}
- good topic extraction

Is brevity an advantage? \cite{Bermingham:2010vh}

Sentiment in twitter events \cite{Thelwall:to}

Distant supervision \cite{Go:2009ut}

\end{comment}

\section{Emotion}
\label{background:emotion}

Defining emotion has been a problem that has puzzled philosophers and thinkers as far back as Cicero and Descartes. Although there is no unifying theory, or completely accepted classification, in general many have agreed there to be two broad types of emotion, the \emph{basic emotions} and \emph{complex emotions}. Basic emotions are biologically innate within all humans, whilst complex emotions are culturally specific amalgamations of our basic one. Deciding upon both what constitutes are basic and complex emotions however has been the cause of significant debate. Perhaps the two most prominent classifications of recent times are those put forward by Ekman \cite{Ekman:1969ux} and Plutchik \cite{Plutchik:2001tp}. 

\subsection{Current defenitions}

After years of work within the field, and having observed the Fore tribesmen of Papua New Guinea, Ekman's 1969 paper presented what he believed to be the six core emotions. These were \emph{anger}, \emph{disgust}, \emph{fear}, \emph{happiness}, \emph{sadness}, \emph{suprise}. Within his work he notes that the Fore tribesmen could identify these emotions when presented with photos of faces expressing them, regardless of their cultural origin. 

In 1980, Robert Plutchik \cite{Plutchik:2001tp} presented his research into human emotion. Within it, he uses five of the emotions put forward by Ekman, whilst introducing three new emotions (see figure \ref{fig:eight_basic_emotions}). Plutchik expands these emotions further, referring to them as \emph{dimensions}, within which different emotions can express varying degrees of their dimension. Furthermore, each of the eight emotion definitions also has a polar opposite definition within the list.

\begin{figure}[h!]
	\caption{Robert Plutchik's eight basic emotions \emph{(* proposed by Ekman)}}
	\label{fig:eight_basic_emotions}
	\centering
	\begin{tabular}{ | l | l | l | l | l |}
		\hline
		Basic Emotion & Polar Emotion & \multicolumn{3}{|c|}{Degrees (strong to weak)}\\
		\hline
		Joy	& Sadness & Ecstasy & Joy & Serenity \\
		Trust	& Disgust & Admiration & Trust & Acceptance\\
		Fear* & Anger & Terror & Fear & Apprehension\\
		Surprise & Anticipation & Amazement & Surprise & Distraction\\
		Sadness* & Joy & Grief & Sadness & Pensiveness \\
		Disgust* & Trust & Loathing & Disgust & Boredom\\
		Anger* & Fear & Rage & Anger & Annoyance\\
		Anticipation* & Surprise & Vigilance & Anticipation & Interest\\
		\hline
	\end{tabular}
\end{figure}

Taking this original list of eight, Plutchik also proposesa further eight complex emotions, formed from combinations of the original eight (see figure \ref{fig:eight-complex-emotions}).

\begin{figure}[h!]
	\caption{Robert Plutchik's eight complex emotions}
	\label{fig:eight-complex-emotions}
	\centering
	\begin{tabular}{ | l | l | l | }
		\hline
		Combined basic emotions & Complex Emotion & Polar Emotion \\
		\hline 
		Anticipation \emph{and} Joy & Optimism & Disappointment \\
		Joy \emph{and} Trust & Love & Remorse \\
		Trust \emph{and} Fear & Submission & Contempt \\
		Fear \emph{and} Surprise & Awe & Aggressiveness \\
		Surprise \emph{and} Sadness & Disappointment & Optimism \\
		Sadness \emph{and} Disgust & Remorse & Love \\
		Disgust \emph{and} Anger & Contempt & Submission \\
		Anger \emph{and} Anticipation & Aggressiveness & Awe \\
		\hline
	\end{tabular}
\end{figure}

The level of detail within Plutchik's research provides a wider scope of definition than that put forward by Ekman. For this reason it shall serve as the classification system we attempt to computationally replicate. Furthermore, Plutchik's proposal introduces concepts of polarity and strength to emotion. Both these concepts bare strong similarities to research within sentiment classification \ref{background:sentiment_classification}, and we will explore the benefits of this similarity in chapter \ref{polarity}.

\subsection{Computational classification}

As with defining emotion, approaches to classifying it are broad and in general, un-unified. Each present their own definitions of emotion, and as a result approaches can vary quite widely. Although essentially a machine learning problem, unlike polarity classification for example, emotion classification can not be attributed to one single field. It is commonly researched within Natural Language Processing, however other fields such as Human-computer interaction, also present their own approaches. 

As with most machine learning problems, classifying emotion can be approached both through supervised and un-supervised techniques. Yang et al. \cite{Yang:2007wx} propose a largely unsupervised approach to classification, largely based upon an emotion corpora built with distant supervision. In a similar fashion to Go et al. \cite{Go:2009ut}, blog posts are scraped for sentences containing emoticons. These sentences are then classified with an emotion based upon the that which Yang et al. perceive the emoticon to hold. Once done, each word in any sentence containing an emotion is given a point-wise mutual information score, based upon how often it co-occurs with its sentence's emotion. With a corpora of words and their relation to any given emotion now built, Yang et al. classify new sentences by examining their corpora to see if any of the sentence's words exist within it. If any words do exist within their corpora, the sentence's score for the word's emotion is increased. Yang et. al see considerable success with this method, however the diversity of the emotion they can label is limited to measures of \emph{valence} and \emph{energy}. Furthermore, the approach uses no manual annotation when testing, instead relying on its unsupervised labels, and thus it could be seen as merely favouring its own error.

In an alternative approach, Alm et al. \cite{Alm:2005vc} put forward a supervised method for classifying emotion. In training their classifier, Alm et al. manually annotate a corpus of 185 children's stories with on of six emotions, \emph{angry}, \emph{disgusted}, \emph{fearful}, \emph{happy}, \emph{sad} and \emph{surprised}. Interestingly, they note that their annotators only achieved an agreement rate of between 45 and 65\%, suggesting that classification rates such as those seen in subjectivity and sentiment analysis can never truly be achieved. With their corpora now annotated, they draw upon thirty core features for training and classifying. Many of these features are similar to those used in polarity classification such as unigrams and adjectives. However, of particular interest and novelty is their use of WordNet as a feature for exploring how the semantic grouping of a word will effect a sentences emotional classification. This is done by grouping WordNet words based upon their relation to each of the six emotions. Although Alm et al. do not report the success rates seen in work by Yang et al., their classifier performs well, on average achieving accuracy rates of 63\% when determining whether emotion exists. They are less successful in classifying the actual emotion and are only able to achieve an F-score of 32\% for when classifying negative emotions such as fear, and 13\% for positive emotions such as happiness. 

It is clear from the work of Alm et al. \cite{Alm:2005vc} that classifying emotion with any degree of detail or certainty is a remarkably difficult process, and one in which there is a lot of progress still to be made. Although emotion classification across a limited range has seen some success, such as that seen by Yang et al., as ones definition of emotion increases in scope, our ability to accurately classify it drastically falls.

\section{Tools and languages}
\label{background:tools}

Although speed is not the primary performance measure within this project, we still hope to maximise efficiency where possible. This desire to strike a balance between programming practicality and speed has defined our choice of tools and languages. As a result languages have primarily been chosen based upon their practicality with regards to rapid development and experimentation, On the other hand, data critical tools such as database software have been chosen with the hope of maximising efficiency whilst remaining flexible. The remainder of this section shall focus upon which languages and tools we have chosen and why, along with any potential pitfalls we may encounter in using them.

\subsection{MongoDB}

MongoDB (www.mongodb.org) is a no-SQL document-oriented database. Essentially, MongoDB discards schemas and relationships, instead leaving us with the complete freedom to structure and organise documents as we choose. Documents (or objects) are stored using \emph{JSON}\footnote{\emph{JavaScript Object Notation is an open standard for defining data structures and objects. Although based upon JavaScript, methods for converting to and from JSON are supported in most languages.}} notation, and can be organised into \emph{collections} of documents. For examples a collection statuses, might consist of a set tweets each stored within MongoDB as JSON objects, such as that in listing \ref{overview:example_json_tweet}.

\begin{lstlisting}[language=Ruby, caption={Example tweet stored within MongoDB}, label=overview:example_json_tweet]
{
	"_id" : ObjectId("4dee4f6697dee90936000082"),
	"text" : "@CalebHowe It wasn't Weiner! http://flic.kr/p/9Rgb84 #weinergate",
	"source" : "twitter_search",
	"source_id" : NumberLong("78133750344597504"),
	"created_at" : "Tue Jun 07 2011 17:18:46 GMT+0100 (BST)",
	"from_user" : "speciallist",
	"from_user_id" : 15966313,
	"to_user" : "CalebHowe",
	"to_user_id" : 8574053
}
\end{lstlisting}

As MongoDB is schema-less, the database makes no structural checks on existing data nor does it check when inserting new data. This means that if there are discrepancies such as additional fields, the database will neither complain or alert the user. 

The simplicity of document-oriented databases has led to a rapid adoption amongst data-driven companies\cite{popescu, moore}, preferring versatility and speed over the benefits of relational databases. With this has come increasing research and stability, ensuring that databases such as MongoDB are not only fast and efficient, but stable and safe. MongoDB was selected for this project in particular, for several reasons:

\begin{itemize}
	\item MongoDB's schema-less no-SQL approach is well suited to changes in our data's structure. Importantly, this means that if Twitter change their API's object model, adaptation is only required within code, rather than in the database as well. Furthermore, additional micro-blogs with new or different properties can be easily introduced and stored within the database.
	\item Due to the lack of a relational layer, when inserting data and performing basic queries, MongoDB performs much faster than SQL derivatives\cite{kennedy}. This is important, both if we hope to retrieve and classify large swathes of data from Twitter, and if we intend to train our classifiers using a large body of data.
	\item Relationships are of little importance to this project, and thus the benefits of a schema-less database far outweigh the negatives.
\end{itemize}

\subsection{Ruby}

Ruby (www.ruby-lang.org) is a high-level interpreted programming language. It was deemed well suited to the project for several reasons:

\begin{itemize}
	\item Its functional aspects make data manipulation simple and fluid. This is particularly relevant for sentiment analysis in which fast data manipulation is essential, especially when building feature sets or partitioning data for training.
	\item Ruby is easily extensible with C/C++, and support for JAVA libraries is stable. This means that we are not simply limited to Ruby libraries when building our classifiers. As most research into sentiment analysis and NLP has been conducted in JAVA or C++, this extensibility is useful in allowing us to make use of leading work and libraries.
	\item Ruby's focus on simplicity means that components can be rapidly developed and experimented with. The freedom afforded by it's simplicity means that we can focus on experimenting with our approach rather than focussing on the complexities of lower-level languages such as C or C++.
	\item Ruby has well-established frameworks for building web-services such as \emph{Sinatra} and \emph{Ruby on Rails}. These allow database-driven websites to be rapidly developed and will enable both simple collection and visualisation of data.
\end{itemize}

This project draws upon some core Ruby frameworks, in particular making use of:

\begin{description}
	\item [Sinatra] is a simple web application framework for Ruby. In essence it allows methods to be easily bound to specific web addresses and is perfect for rapidly developing sites. It is stable and well supported by the open source community. Within this project it shall be used to deliver the data visualisation and provide a backend through which tweets can be found and labelled in order to train our classifiers.
	\item [Artificial Intelligence for Ruby] is a detailed library providing an array of machine learning focussed tools. In particular the library includes a Naive-Bayes classifier which will be used within each of our three classifiers.
	\item [LIBSVM] is a well established C++ library for support vector machines. Originally proposed and built by Chang et al. \cite{Chang:2001ta}, it has become a popular tool for SVM classification. This will be used within Ruby through a LIBSVM interface written by Tom Zeng\footnote{https://github.com/tomz/libsvm-ruby-swig}. As LIBSVM is written in C++ it is much faster than Ruby alternatives. Additionally it supports multi-class classification without any additional work.
	\item [MongoMapper] is a Ruby framework for interfacing between MongoDB and Ruby. It provides an array of functionality, and in particular simplifies the task of building Ruby classes which map to MongoDB documents. Rather than fetching data from MongoDB and storing it in a Ruby object, MongoMapper creates an open link between the two. Thus whenever attributes are read or written to a Ruby object, they are in fact directly read or written to the corresponding MongoDB document.
\end{description}

\subsection{Processing.js}


\chapter{Code examples}

\section{Content retrieval}

\begin{lstlisting}[language=Ruby, caption={Illustration of Status class' MongoMapper attributes}, label=retrieval:status_code]
class Status
  include MongoMapper::Document

  # Attributes
  key :text, String         
  key :source, String
  key :source_id, Int
  key :posted_at, DateTime
	key :from, String

  # Relationship attributes
  key :classified_status, ClassifiedStatus
  key :trained_status, TrainedStatus 
end
\end{lstlisting}

\begin{lstlisting}[language=Ruby, caption={Example Twitter search API results}, label=retrieval:example_search]
{
	"results":[
		{
			"text":"@twitterapi, look at my example tweet!",
			"to_user_id":396524,
			"to_user":"TwitterAPI",
			"from_user":"jkoum",
			"metadata":
			{
				"result_type":"popular",
				"recent_retweets": 100
			},
			"id":1478555574,
			"from_user_id":1833773,
			"iso_language_code":"nl",
			"profile_image_url":"http://twitter.com/image.jpg",
			"created_at":"Wed, 08 Apr 2009 19:22:10 +0000"
		}
	],
	"since_id":0,
	"max_id":1480307926,
	"refresh_url":"?since_id=1480307926&q=%40twitterapi",
	"results_per_page":15,
	"next_page":"?page=2&max_id=1480307926&q=%40twitterapi",
	"completed_in":0.031704,
	"page":1,
	"query":"%40twitterapi"
}
\end{lstlisting}

% \lstinputlisting[language=Ruby, caption={Example Twitter search API results}, label=retrieval:example_search]{../../lib/sentiment/statuses/status.rb}
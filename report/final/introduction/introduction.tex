% This is one of the most important components of the report. It should begin with a clear statement of what the project is about so that the nature and scope of the project can be understood by a lay reader. It should summarise everything you set out to achieve, provide a clear summary of the project's background, relevance and main contributions. It should explain the motivation for the project (i.e., why the problem is important) and identify the issues to be addressed (i.e., why the problem is difficult). The introduction should set the scene for the project and should provide the reader with a summary of the key things to look out for in the remainder of the report. When detailing the contributions it is helpful to provide pointers to the section(s) of the report that provide the relevant technical details. The introduction itself should be largely non-technical. It is sometimes useful to state the main objectives of the project as part of the introduction. However, avoid the temptation to list low-level objectives one after another in the introduction and then later, in the evaluation section (see below), say something like "All the objectives of the project have been met blah blah...". A project that meets all its objectives is, by definition, weak and unambitious. Concentrate instead on the big issues, e.g. the main questions (scientific or otherwise) that the project sets out to answer.

\chapter{Introduction}

% The ability to listen to and understand public opinion serves as the foundation upon which democratic society is built. Although the merits of different electoral systems can be debated, in general they all provide effective and functional methods for democratically electing governments. 

\section{Motivation}

From women's rights to civil rights, the influence of public opinion on government policy has been pivotal. Within a healthy democracy the voice of the electorate should be heard and recognised by those chosen to represent them. Throughout history platforms have often been provided for public opinion to be made known, from the early public forums of Greece and Rome, to Speaker's Corner and the House of Commons today. Providing a means for people to express their opinion enables them to both challenge and shape the direction their elected governments take. As a result, finding ways of gathering and understanding this opinion has increasingly proved fundamental if a government wishes to be successful.

Current methods of measuring opinion are largely statistical, with methods such as polling looking at the opinion of a sample group, before using their results to make further predictions. These can be informative, however their small sampling rates mean that figures can often be askew. Furthermore polling is both costly and time consuming to conduct, and thus can be used neither to find opinion on breaking news, nor across a large number of topics. Nonetheless, as methods for measuring public opinion have improved both in accuracy and detail, politicians and policy makers have increasingly turned to them not only for affirmation of their policies, but for guidance and new initiatives.

As the web has become more prevalent throughout society, it is increasingly becoming a platform for discussion and opinion. The initial growth of blogging demonstrated the web's ability to serve as a forum for debate and opinion. However the technical knowledge required to start a blog, alongside the time required to write a post meant that adoption was limited. In the past two years we have seen the rise of micro-blogging (essentially 140 character blog posts) through services such as Twitter. These have seen unprecedented levels of adoption, with Twitter's 200 million users posting 25 billion micro-blog posts in 2010. Due to the simple nature of writing short posts, micro-blog discussion tends to break quickly around news topics, and offers genuine insight into public opinion surrounding news topics.

This project hopes to utilise the growth of publicly available opinion on the web, using it as a source upon which new methods for analysing and measuring public opinion can be built. In particular the project will focus on understanding sentiment on micro-blogging services such as Twitter. 

\section{Contributions}

Based upon this, our project hopes to deliver the following core set of contributions,
\begin{itemize}
\item An engine for collecting and understanding sentiment on Twitter:
	\begin{itemize}
		\item A Twitter scraper which will fetch live statuses from Twitter.
		\item A method for determining whether fetched statuses are opinionated or not.
		\item A method for identifying whether fetched statuses' opinion is positive or negative.
		\item A method for identifying the core topics which our fetched statuses are discussing.
		\item A means of persisting our statuses and the results of the above methods, such that they can be easily retrieved at any time.
	\end{itemize}
\item A web service which can serve as an interface to our engine, through which,
	\begin{itemize}
		\item Requests can be made to classify a status, upon which the service should return the classification engine's results.
		\item A method for requesting any classified data our engine may have stored.
	\end{itemize}	
\end{itemize}

Combined, these should provide a tool for collecting, analysing and storing sentiment from Twitter. We should be able to read and access this in order to gain a better understanding of overall sentiment on Twitter. The web service should make our data available to others so they to can use it however they deem fit.

Additionally, if possible we hope to implement the following extensions to our project,
\begin{itemize}
	\item Extend our sentiment analysis engine to give a more detailed account of a broader spectrum of emotion.
	\item Web based visualisation tool for better understanding and exploring sentiment on Twitter
\end{itemize}